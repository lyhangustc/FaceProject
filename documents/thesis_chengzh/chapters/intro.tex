% !TeX root = ../main.tex

\chapter{绪论}

\section{研究背景}

%\subsection{二级节标题}
%
%\subsubsection{三级节标题}
%
%\paragraph{四级节标题}
%
%\subparagraph{五级节标题}

人脸图像生成一直是计算机图形和视觉领域的热门方向,包括人脸姿态仿真\cite{Yin_2017_ICCV}、遮挡人脸恢复\cite{Li_2017_CVPR}、年龄与表情仿真\cite{G_2017_ICIP,ggfr}、面部属性编辑\cite{attrgan}、人脸艺术生成\cite{carigan}、人脸美颜与风格化\cite{beautygan}等等。而基于草图的人脸图像生成是这一方向的重要分支。

当我们想要从零开始描绘一个物体时,手绘草图无疑是最直观、最有效的方式之一,它比文字描述更清晰具象,可以将抽象的概念转化为具体的视觉表达。随着触屏设备的广泛普及,随手绘制草图变得越来越简单容易。所以如何由手绘草图生成真实的图像一直吸引着领域内研究人员的关注。由手绘草图生成真实的人脸有着广泛的应用。比如在刑侦案件中,根据目击者的手绘草图重塑嫌疑人的人脸图像;还可以根据我们自己的兴趣,随心所欲地将手绘的人脸草图转化成真实的人脸照片等等。而且,草图有着很强的通用性,是一种跨越国家和文化的沟通交流方式,不同国家和民族的人们都可以通过手绘草图描绘自己的所见所闻,表达自己的所思所想。草图简单明了,但却可以捕捉对象的结构特征和纹理细节,包含了足够的信息,可以作为有效的输入来生成高质量的图像。

经过近二十年的发展,由草图生成真实图像的方案主要有两种。一种是传统的方法,图像检索。该方法需要基于一个庞大的数据库,将手绘草图分割成独立的部分,利用搜索算法分别检索出与输入的草图相匹配的图像,然后将所有部分进行融合,生成一张完整的图像。该方法缺点很明显,那就是需要大量的数据,保证数据库中有与输入草图相匹配的内容,并且后期的融合会有很明显的痕迹。这种方法显然不适合人脸图像的生成。

另一种是基于深度学习的方法,图像翻译\cite{pix2pix}。图像翻译的目标是给定输入-输出图像对作为训练数据,将输入图像从一个域$\mathscr{A}$(源域)转换到另一个域$\mathscr{B}$(目标域)。如果用$\alpha (x, \symup{A}) \in \mathscr{A}$和$\beta(x,\symup{B})\in\mathscr{B}$来表示域$\mathscr{A}$和域$\mathscr{B}$的图像,图像翻译的任务可以描述为,寻找一个合适的变换$f:\mathscr{A}\to\mathscr{B}$,使得
\begin{equation}
  f(\alpha (x, \symup{A})) = \beta(x,\symup{B})
\end{equation}

该方法基于生成对抗网络GAN\cite{gan},并且是以图像为输入的条件生成对抗网络\cite{cgan}。在训练阶段,需要大量成对的数据作有监督的学习。条件生成对抗网络的网络结构分为两部分,即生成器和判别器。训练时将草图输入生成器,生成一张图像,然后将生成的图像与真实的图像分别输入判别器,以判别器的输出代入损失函数,分别计算生成器与判别器的损失,通过后向传播算法更新网络参数,使生成器产生更接近真实图像的输出。图像翻译的概念最早由朱俊彦在pix2pix\cite{pix2pix}这篇论文中提出,而后又在此基础上发表了另一篇文章pix2pixHD\cite{pix2pixhd},可以生成质量更高的图像。本文的主要工作就是围绕pix2pixHD展开。

\section{本文工作}

本文旨在探究利用pix2pixHD实现从草图到人脸照片生成的过程,发现对于手绘草图的输入,其输出结果的质量不够令人满意,特别是某些细节比较模糊,而且草图上某一位置的改变常引起生成结果全局的变化。针对这一缺陷,我们用可视化的手段分析了草图特征,对其网络结构进行了深度剖析,在此基础上改造了其生成器的网络结构,我们改进后的模型对手绘草图输入的生成图像质量更高,并且能更好地实现图像编辑的功能。

首先,本文用人脸轮廓-照片数据对训练pix2pixHD的原始模型,然后用手绘的草图进行测试,发现其生成的人脸照片不尽人意,具体表现为纹理特征不够清晰,丧失了对细节的控制,如果在草图上改动一处可能影响生成的整体结果。

而后,利用pca、tsne等可视化工具对生成器网络提取到的草图特征进行可视化分析,结合对网络结构的研究,本文发现生成器编码图片所用的实例标准化层是造成这一现象的根本原因。接下来我们针对生成器中的实例标准化层进行了消融实验,做对比研究,发现去掉前两层实例标准化操作的模型生成质量更高,且能更好地对输入草图进行编辑。

针对模型对于输入草图空间位置变化的鲁棒性差的问题,我们用增广的数据重新训练了模型,成功解决了这一问题。

%\section{脚注}
%
%Lorem ipsum dolor sit amet, consectetur adipiscing elit, sed do eiusmod tempor
%incididunt ut labore et dolore magna aliqua.
%\footnote{Ut enim ad minim veniam, quis nostrud exercitation ullamco laboris
%  nisi ut aliquip ex ea commodo consequat.
%  Duis aute irure dolor in reprehenderit in voluptate velit esse cillum dolore
%  eu fugiat nulla pariatur.}
