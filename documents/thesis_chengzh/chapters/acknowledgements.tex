% !TeX root = ../main.tex

\begin{acknowledgements}

经过这几个月来的学习和调研、思考和实验,我终于完成了自己的本科毕业设计,此刻我思绪万千,想说的话有很多,但最想表达的是自己满怀的感激。

首先,我要感谢的人就是我的导师陈雪锦副教授。每当我遇到困惑不解的问题时,她总能给我提供宝贵的指导意见,使我拨云见日、柳暗花明。每当我放松懈怠时,她又会及时地点醒我,鞭策我不断向前。在她的引导下,我学会更加深入地思考而不只流于问题的表面。她还亲自动手帮我修改论文,叮嘱我写作论文的注意事项,令我感动不已。老师对我的教诲、包容与鼓励,给了我克服困难、直面挫折的信心和勇气,使我能顺利完成自己的毕业设计。老师严谨的治学态度和孜孜以求的探索精神,给我以后的科研道路树立了榜样,也让我获益匪浅。

其次,我要感谢科技西楼1205多媒体计算与通信实验室的所有同学们,特别是组内的李宇航、杨彬鑫和陈梓涵三位师兄,在我完成毕设的过程中给予我很多指导与帮助。正是他们的倾囊相授与无私帮助,才能让我在基础薄弱、经验缺乏的情况下一步步完成本文的工作。

我还要感谢我的班主任唐建老师,在大学四年里给予我很多关怀与帮助;感谢辅导员袁国富老师,多少次的促膝谈心、谆谆教诲,在工作和学习中给我支持与鼓励;感谢我的三位室友以及班里所有的同学们,陪伴我走过大学四年的时光,让我的青春焕发光彩。

同时,我还要感谢我的父母和爷爷奶奶,他们用无私的爱点亮了我的生活,尽他们所能给我营造最好的环境,但凡我取得一点点进步都离不开他们的奉献与付出。他们是我求学路上不断进取的动力,是我的信心之源。

2020年注定是个多事之秋,我们国家刚刚经历过大疫,我们每个人也都被迫在家不能返校。但是我多想在毕业之前再回去见一面朝夕相处了四年的同学,说不定自兹之后各自天涯,我多想把毕业这一刻的喜悦跟母校分享,将毕业这一刻的美好定格于母校的每个角落。

所以最后,我要感谢我的母校,中国科大,在这里我汲取知识,丰富自我,结交了许多良师益友,度过了青春最珍贵的四年时光,这对我的一生都意义深远。我会继续努力,不断求索,不辜负母校对我的培养,也希望母校能越办越好,创寰宇学府纳天下英才,早日成为世界一流大学。

\end{acknowledgements}
