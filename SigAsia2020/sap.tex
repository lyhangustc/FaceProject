\subsubsection{Spatial Attention Pooling}
When the input hand-drawn sketch is not well-drawn, it is a trade-off between the realism of the output face image and the alignment between the input sketch and the output face image.
%
In order to alleviate the edge alignment between the input sketch and the output face image, we should relax the sharp sketch lines with one-pixel width.
One of the straightforward ways is to smooth the lines of sketches using image smoothing algorithm. 
Another is to dilate the sketch lines so that the widths of lines are of multiple pixels~\cite{DeepSurgery}.
However, the capacity of either the two hand-craft ways above is limited. Because the smoothness and the dilate radius are the same for all positions of the whole image.
%
We argue that the balance between the realism of the output face image and the alignment between the input sketch and the output face image differs from one position to another across the face image. Therefore, the smoothness or the dilation radius should be spatial-specific. 

Based on the discussion above, we propose a new module, called spatial attention pooling (SAP), to dilate the input sketch in a spacial-specific way. Let $\mathbf{r}=\{r_i | i=1,2,...,N_r\}$ be a set of dilation radius. Given an input sketch $s\in \real^{H\times W}$, we first pass it through $N_r$ pooling branches with kernel sizes of $\mathbf{r}$ to get $\{P_{r_i}(s) | i=1,2,...,N_r\}$. Then we compute the spatial attention map $W\in \real^{N_r\times H\times W} $with $W = Softmax(f(s))$, where $f()$ is implemented with two convolutional layers. A softmax layer which is computed over channels is added at the end of the convolutional layers, ensuring that for each position, the sum of weights of all channels equals to $1$. The output of SAP is computed as:
	
\begin{equation}
	SAP(s)=\sum_{i=1}^{N_r} W_i * P_{r_i}(s),
\end{equation}

where $W_i$ is the $i$th channel of $W$.