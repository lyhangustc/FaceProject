% !TeX root = SketchFace.tex
% 



%\subsection{Model Overview}
%\label{subsec:algorithm_overview}

In this paper, we propose a sketch-to-photo translation model that is robust to hand-drawn sketches.

The architecture of the proposed model is shown in Figure~\ref{fig:architecture}. We use pix2pixHD model~\cite{pix2pixHD} with the 'global' version generator and the multi-scale discriminator as baseline model. The input of the generator is either a face sketch or a deformed face sketch and the output is the corresponding generated photo-realistic face image. A novel module named Spatial Attention Pooling (SAP) is added in the front of the generator to enable our model to adaptively handle line distortions of input sketches \td(before the lowest level of feature extraction.) 
%
For the discriminator, the generated image concatenated with the input sketch (or the deformed input sketch) is treated as a fake sample while a real face image sampled from real face distribution concatenated with its corresponding sketch is regarded as a real sample. 
%
%In the adversarial training stage, the generator and the discriminator are updated alternately.
%
In order to guided the model with SAP to be tolerant with line distortion of input sketches, we design a novel generator feature matching loss for our task, besides the adversarial loss, the reconstruction loss and the discriminator feature matching loss used in pix2pixHD. The model is trained in a multi-stage training schedule.
%

In this section, we first introduce the dataset we use in our experiment in Subsection~\ref{subsec:algorithm_}. Then we describe the proposed SAP in Subsection~\ref{subsec:algorithm_sap}. At last we discuss losses applied in our model in Subsection~\ref{subsec:algorithm_loss} and the multi-stage training schedule in Subsection~\ref{subsec:training}.

