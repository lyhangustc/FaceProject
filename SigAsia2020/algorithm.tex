\section{Deep Network for Sketch-Photo Translation}
\label{sec:network}

\subsection{Edge alignment in baseline Model}

Generating a realistic photo from sketch can be considered as an image translation problem. 
We use the state-of-the-art Pix2PixHd~\cite{} as our baseline. 
% 
 
We use the CelebA-HD dataset~\cite{} which contains \td{xxx face images in WxH}. All the face images are globally aligned according to their eye positions. 
For each photo, we generate sketch samples \td{by xxx method}.
\td{XX for training and xx for testing.}
%
Using this paired sketch-photo dataset, \td{we trained our method and other state-of-the-art approaches for comparison}.

\subsubsection{Global alignment}
While all the face images in the CelebA dataset \cite{} are globally aligned with their eye positions, the learned generator implicitly embeds the global layout. Once the drawn sketches deviate from this implicitly embedded layout, the learned translation model generate awkward results, as Fig.~\ref{fig:global-align-fail} shows. 


\begin{figure}
	\centering
	\vspace{1.0cm}
	\caption{Generated face images from a sketch that does not follow the aligned face layout.}
	\label{fig:global-align-fail}
\end{figure}

\subsubsection{Data augmentation with geometric translation}
A straightforward way is to augment the training set by random geometric transformation of input sketches. 
However, large interval/range of the transform yields un-convergence of the network training. 
We limit the transformation range to $(-\frac{\pi}{10},\frac{\pi}{10})$ rotation, $(-\frac{W}{20},\frac{W}{20})$ translation, and \cxj{scale?} in order to increase the tolerance of the trained model on the distortion and roughness of hand-drawn sketches. 
%
Moreover, as an interactive system, we also simply provide a reference sketch, like ShadowDrawing~\cite{}.
We place an averaged face contour image on the canvas to provide a reference coordinate system for user to draw their strokes. 
Therefore, the drawn sketches under this geometric reference will be located in or close to space of the training sketches.

\subsubsection{Data augmentation with sketch dilation}

\subsubsection{pix2pixHD without instance normalization}
The baseline model, as well as many existing stylization DNNs, uses spatial normalization to extract style statistics, treating them as spatially uniform on the entire image. 
However, the shallow convolution layers typically extract texture or brightness statistics, which are varying dramastically on different local regions of a drand-drawn sketch. For example, there might be a large number strokes around hair regions or mouth regions to describe details. These heavy strokes bring significant difference while instance normalization at each convolution layer normalizes local patch features with a global factor. Therefore, the extracted features for identical eye strokes could be very different. 

We think that strokes mainly describe the shape features, without little texture information. 

\subsubsection{Spatial Attention Pooling}
When the input hand-drawn sketch is not well-drawn, it is a trade-off between the realism of the output face image and the alignment between the input sketch and the output face image.
%
In order to alleviate the edge alignment between the input sketch and the output face image, we should relax the sharp sketch lines with one-pixel width.
One of the straightforward ways is to smooth the lines of sketches using image smoothing algorithm. 
Another is to dilate the sketch lines so that the widths of lines are of multiple pixels~\cite{DeepSurgery}.
However, the capacity of either the two hand-craft ways above is limited. Because the smoothness and the dilate radius are the same for all positions of the whole image.
%
We argue that the balance between the realism of the output face image and the alignment between the input sketch and the output face image differs from one position to another across the face image. Therefore, the smoothness or the dilation radius should be spatial-specific. 

Based on the discussion above, we propose a new module, called spatial attention pooling (SAP), to dilate the input sketch in a spacial-specific way. Let $\mathbf{r}=\{r_i | i=1,2,...,N_r\}$ be a set of dilation radius. Given an input sketch $s\in \real^{H\times W}$, we first pass it through $N_r$ pooling branches with kernel sizes of $\mathbf{r}$ to get $\{P_{r_i}(s) | i=1,2,...,N_r\}$. Then we compute the spatial attention map $W\in \real^{N_r\times H\times W} $with $W = Softmax(f(s))$, where $f()$ is implemented with two convolutional layers. A softmax layer which is computed over channels is added at the end of the convolutional layers, ensuring that for each position, the sum of weights of all channels equals to $1$. The output of SAP is computed as:
	
\begin{equation}
	SAP(s)=\sum_{i=1}^{N_r} W_i * P_{r_i}(s),
\end{equation}

where $W_i$ is the $i$th channel of $W$.





\paragraph{Sketch Data Generation}
Four options to generate sketches: Canny, HED, AutoTrace [Web..], and Contour. 

. 