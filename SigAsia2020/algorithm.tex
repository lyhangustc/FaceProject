% !TeX root = SketchFace.tex

\section{Deep Network for Sketch-Photo Translation}
\label{sec:network}

\subsection{Edge alignment in baseline Model}

Generating a realistic photo from sketch can be considered as an image translation problem. 
We use the state-of-the-art Pix2PixHd~\cite{} as our baseline. 
% 
 
We use the CelebA-HD dataset~\cite{} which contains \td{xxx face images in WxH}. All the face images are globally aligned according to their eye positions. 
For each photo, we generate sketch samples \td{by xxx method}.
\td{XX for training and xx for testing.}
%
Using this paired sketch-photo dataset, \td{we trained our method and other state-of-the-art approaches for comparison}.

\subsubsection{Global alignment}
While all the face images in the CelebA dataset \cite{} are globally aligned with their eye positions, the learned generator implicitly embeds the global layout. Once the drawn sketches deviate from this implicitly embedded layout, the learned translation model generate awkward results, \cxj{as Fig.~\ref{fig:global-align-fail} shows}. 


\begin{figure}
	\centering
	\vspace{1.0cm}
	\caption{Generated face images from a sketch that does not follow the globally aligned face layout.}
	\label{fig:global-align-fail}
\end{figure}

\subsubsection{Data augmentation with geometric translation}
A straightforward way is to augment the training set by random geometric transformation of input sketches. 
However, large interval/range of the transform yields un-convergence of the network training. 
We limit the transformation range to $(-\frac{\pi}{10},\frac{\pi}{10})$ rotation, $(-\frac{W}{20},\frac{W}{20})$ translation, and \cxj{scale?} in order to increase the tolerance of the trained model on the distortion and roughness of hand-drawn sketches. 
%
Moreover, as an interactive system, we also simply provide a reference sketch, like ShadowDrawing~\cite{}.
We place an averaged face contour image on the canvas to provide a reference coordinate system for user to draw their strokes. 
Therefore, the drawn sketches under this geometric reference will be located in or close to space of the training sketches.

%%%% \subsubsection{Data augmentation with sketch dilation}
% !TeX root = SketchFace.tex


Paired face sketch-photo dataset is required for supervised sketch-to-face translation methods.
Since there exits no large-scale paired face sketch dataset, the training face sketches used by existing methods are generated from face image dataset, e.g. CelebA-HQ face dataset, using edge detection algorithm such as HED~\cite{HED}.
%
However, the sketches generated by edge detection algorithm are sometimes incomplete or \td{other problems}. 
\td{Discuss the advance of make-edges over edge maps and contours}
\cxj{I would say the edge maps are quite different from handdrawn sketches, not because of the incompleteness.}
\rmv{ Although CSAGAN~\cite{CSAGAM} applied self-attention mechanism to alleviate the incompleteness problem, the others remains. Therefore we use another method to generate clearer and complete sketches from face images.}

~\cite{pix2pixHD} introduce another method to generate sketches from face images. Given a face image, the face landmarks are detected using an off-shelf landmark detection model. A new kind of sketch, denoted as \textit{face contour}, is obtained by connect specific landmarks. Sketch-to-face model trained by face contour fail to generalize to hand-drawn sketches with hair, wrinkles or beard. 

The CelebAMask-HQ dataset~\cite{CelebAMask-HQ} provides a face semantic map for each face image in CelebA-HQ dataset. We basically use the boundary map of the semantic map as the sketch of the corresponding face image. Figure~\ref{fig:sketch_data} shows an example of comparison between a sketch generated by edge detector, a face contour and a sketch generated from semantic boundary.
%


\paragraph{Stroke Deformation}
A shortcut of sketches generated from semantic boundary (and those generated by edge detector) is the lines of sketches are perfectly aligned to edges of the corresponding face images. In order to break the edge-alignment and mimic the hand-drawn sketches, we apply a deformation to the lines, similar to that in FaceShop~\cite{FaceShop}. Specifically, we vectorize lines of each sketches using AutoTrace algorithm~\cite{AutoTrace}, and add an offset randomly selected from $[-d, d]^2$ to the control points and end points of the vectorized lines, where $d$ is the maximum offset and we set $d=11$ in our experiments.
%
Both the initial sketches generated from semantic boundary and deformed sketches are used as the input sketches to our model.


\begin{figure}
	\centering
	\vspace{1.0cm}
	\caption{Comparison between a sketch generated from edge detection and from semantic boundary.}
	\label{fig:sketch_data}
\end{figure}

\subsubsection{pix2pixHD without instance normalization}
The baseline model, as well as many existing stylization DNNs, uses spatial normalization \cxj{instance normalization?} to extract style statistics, treating them as spatially uniform on the entire image. 
The instance normalization is defined as
\begin{equation}\label{eq:instance-norm}
f^i(x,y,c)=\frac{f^i(x,y,c)}{\sum_{x,y}f^i(x,y)},
\end{equation}
where $i, x,y,c$ respectively indicates feature map layer, pixel position, and channel indexes.  
However, the shallow convolution layers typically extract texture or brightness statistics, which are varying dramatically on different local regions of a drand-drawn sketch due to its sparsity. For example, there might be a large number strokes around hair regions or mouth regions to describe details. These heavy strokes bring significant difference while instance normalization at each convolution layer normalizes local patch features with a global factor. Therefore, the extracted features for identical eye strokes could be very different. 
This instance normalization term leads to significant change of local shapes, as shown in Fig.~\ref{}.\cxj{add a figure here or show comparison in result section. }


We think that strokes mainly describe the shape features, without little texture information. Normalization is expected to remove the variance of global stroke styles, such as stroke widths, stroke curveness, and stroke coarseness. 
During painting, the stroke coarseness is spatially varying when user depict different facial features. 
The user may draw very carefully for features such as eyes, eyebrows, but not so accurate on cheek shapes. 



%%% \subsubsection{Spatial Attention Pooling}
% !TeX root = SketchFace.tex

 
A sketch-to-image model trained with edge-aligned sketch-image pairs tends to generate images whose edges strictly align with the stokes of the input sketch.
When an input hand-drawn sketch is not well-drawn, line distortions in the input sketch damages the quality of the generated face image. 
It is a trade-off between the realism of the generated face image and the correspondence between the input sketch and the output face image.
%
In order to alleviate the edge alignment between the input sketch and the output face image, we propose to relax thin strokes to a tolerance region with various width.
%
A straightforward way is to smooth the strokes to multi-pixel width by image smoothing or dilation. 
%
However, the capacity of this hand-crafted way is limited, because the uniform smoothness for all positions of the whole sketch violate the unevenness of hand-drawn sketches on depicting different facial parts. 
%
We argue that the balance between the realism and the correspondence differs from one position to another across the face image. Therefore, the relaxation degree should be spatially varying. 


Based on the discussion above, we propose a new module, called spatial attention pooling (SAP), to adaptively relax the strokes in the input sketch to spatially varying tolerance regions. 
%
A stroke with a larger width indicates the less restrict between this stroke and the corresponding edge in the synthesized image. The widths are controlled by the kernel sizes of pooling operators. However, the kernel size of pooling operator is not trainable using back propagation algorithm. SAP applies multiple branches of pooling operators with different kernel sizes to get multiple relaxed sketches with different widths. The relaxed sketches are then fused by a spatial attention layer which spatially adjusts the balance of \textit{realism} and \textit{correspondence}. The module is formulated as follow.

%Let $\mathbf{r}=\{r_i | i=1,2,...,N_r\}$ be a set of relax radius. 
The architecture of SAP is shown in Figure~\ref{fig:sap}.
Given an input deformed sketch $S'\in \real^{H\times W}$, we first pass it through $N_r$ pooling branches with different kernel sizes of $\{r_i, i=1,\ldots, N_r\}$ to get $\P_{i}=Pooling_{r_i}(S') (i=1,\ldots,N_r)$. 
Then we utilize convolutional layers to extract feature maps of $P_i$ separately. These feature maps are concatenated to get a relaxed representation of $S'$, denoted as $R$:
%
\begin{equation}
R=Cat(Conv_1(P_1), Conv_2(P_2),..., Conv(P_{N_r})),
\end{equation}
where $Conv_i() (i=1,\ldots,N_r)$ indicates convolutional layers, $Cat$ is a channel-wise concatenate operator.

On the other hand, we compute a spatial attention map $A$ which controls the relax degrees of all positions by assigning different attention weights to $R$.
A stoke with a large distortion is supposed to be assigned with a large relax degree. Hence, $A$ is supposed to adaptively pay more attention (a large weight) to a $Conv_i(P_i)$ with a large kernel size in the areas with large line distortions.
%
A straightforward way to get $A$ is passing the input sketch through a few convolutional layers and these convolutional layers are trained to detect the areas with line distortions. However, we found the a few convolutional layers are insufficient to learn to detect line distortions directly. Therefore, we introduce a two-class classifier to ease the detection. Specifically, we pre-train a fully-convolutional two-class classifier $C$ with three convolutional layers to distinguish sketches from deformed sketches. Then we utilize this pre-trained classifier to extract features of the input sketch $S$ to get ${C_i(S), i=1,2,3}$, where $C_i(·)$ denotes the $ith$ feature maps extracted by $C$. These feature maps from classifier emphasize the differences between sketches and deform sketches. We resize and concatenate these feature maps, and pass them through three convolutional layers to get the spatial attention map:
%	
\begin{equation}
A=Softmax(Conv([C_1, Up_2(C_2), Up_4(C_3)])), 
\end{equation}
%
where $Up_2$ and $Up_4$ indicates $2\times$ and $4\times$ upsampling, Conv(·) indicates three cascaded convoutional layers, and $Softmax(·)$ is a softmax layer computed over channels to ensuring that for each position of $A$, the sum of weights of all channels equals to $1$.

At last, the output SAP is computed as:
%	
\begin{equation}
SAP(S')=A*R,
\end{equation}
%
where $*$ is element-wise multiplication.

%







%%% \subsubsection{Spatial Local Respond Normalization}
%\input{slrn}
%%% \subsubsection{Losses}
% !TeX root = SketchFace.tex

Our model consists of two generators, $G()$ for edge-aligned sketch $S$ and $G'()$ deform sketch $S'$, and one discriminator $D()$. Loss functions and objective of our model are discussed as follow.

\paragraph{Reconstruction Loss}
For either generator, a reconstruction loss is applied to guide the generated image to get close to its corresponding real image $x$.

\begin{equation}
\label{eqn:loss_rec}
\begin{aligned}
\mathcal{L}_{Rec}(G, G') &=\mathbb{E}_{(S, x)\sim p_{data}(S,x)\|G(S) - x\|_1} \\
&+ \mathbb{E}_{(S', x)\sim p_{data}(S',x)\|G'(S') - x\|_1},
\end{aligned}
\end{equation}

\paragraph{Adversarial Loss}
The multi-scale discriminator~\cite{pix2pixHD} $D$ consists of three sub-discriminators $D_i, i=1,2,3$.  The adversarial loss for $G$ and $D$ is defined as:
\begin{equation}
\label{eqn:new_loss_adv}
\begin{aligned}
\mathcal{L}_{adv}(G;D)&=\frac{1}{3}\sum_{i=1}^{3}E_{(\bm{S},\bm{x})\sim p_{data}(\bm{S},\bm{x})}\big[\log D_i(\bm{S},\bm{x})\big] \\
& + E_{\bm{x}\sim p_{data}(\bm{S})}\Big[\log \Big(1-D_i \big(\bm{S},G(\bm{S})\big)\Big)\Big].
\end{aligned}
\end{equation}
The adversarial loss for $G'$ and $D$ is defined similarly.

\paragraph{Discriminator Feature Matching Loss} Similar to pix2pixHD\cite{pix2pixHD} and lines2face~\cite{Lines2Face}, we use a discriminator feature matching loss as the perceptual loss, which is designed to minimize the error between generated image and real image in feature space. Here discriminator feature matching loss use the discriminator as the feature extractor. Let $D^q_i()$ be the output of $q$th layer in $D_i$. This loss for $G$ and $D$ is defined as:
\begin{equation}
\label{eqn:feature_matching_loss}
\mathcal{L}_{fm}(G)=\frac{1}{3N_Q}\mathbb{E}_{(\bm{S},\bm{x})\sim p_{data}(\bm{S},\bm{x})}\sum_{i=1}^{3}\sum_{q\in Q} \frac{1}{n_i^q} \|D^q_i(G(\bm{S}))-D^q_i(\bm{x})\|_1 ,
\end{equation}
where $Q$ is the selected layers of discriminator for computing this loss, $NQ$ is the number of elements in $Q$, $n^q_i$ is the number of elements in $D^q_i$.
Also, the discriminator feature matching loss for $G'$ and $D$ is defined similarly.

\paragraph{Generator Feature Matching Loss}
Similar to discriminator feature matching loss which is designed to minimize the error between generated image and real image in feature space, we propose a novel generator feature matching loss that aims to minimize the error between the presentations of $S$ and $S'$ in generator feature space. Let $G^t()$ and This loss is calculated as:

\begin{equation}
\label{eqn:loss_GFM}
\mathcal{L}_{GFM}(G, G')=\mathbb{E}_{(S, S')\sim p_{data}(S, S')} \frac{1}{N_T} \sum_{q\in Q}  \frac{1}{|G_q|} \|G_q(S)-G_q(\tilde{S}) \|_1,
\end{equation}


Let $G_q(\cdot)$ produces the feature maps of the $q$-th layer in the generator $G$.
%
Given an input sketch $S$ and the corresponding deformed sketch $\tilde{S}$, we compute the generator feature matching loss as:


%
where $|G_q|$ denotes the number of elements in $G_q(\cdot)$, $Q$ indicates a set of the selected generator layers for computing this loss and the size of $Q$ is $N_Q$. 
We select \td{the xxx layers} of the generator in our experiments.

Besides the generator feature matching loss $\mathcal{L}_{GFM}(G)$, for generator $G$ and multi-scale discriminator $D={D_k | k=1,2,...,N_D}$, the adversarial loss $\mathcal{L}_{GAN}(G, D)$ and the discriminator feature matching loss $\mathcal{L}_{DFM}(G, D)$ are computed as the same form as those in pix2pixHD~\cite{pix2pixHD}. \td{Discussion: add equations of these two losses or not}
%
The objective of the proposed model is:

\begin{equation}
	\label{eqn:new_minmax_game}
	\min_G \max_{D} \mathcal{L}_{GAN}(G, D)+\lambda \mathcal{L}_{DFM}(G, D) +\mu \mathcal{L}_{GFM}(G).
\end{equation}

where $\lambda$ and $\mu$ are the weights for balancing different losses. We set $\lambda=\td{xxx}$ and $\mu=\td{xxx}$ in our experiments.




%\paragraph{Sketch Data Generation}
%Four options to generate sketches: Canny, HED, AutoTrace [Web..], and Contour. 

