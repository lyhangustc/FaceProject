% !TeX root = SketchFace.tex

\subsubsection{Losses}
\label{subsec:algorithm_loss}

\td{generator feature matching effect and losses summary}

\cxj{Before describe how you do it, please describe why you do this. }

Let $G_q(\cdot)$ produces the feature maps of the $q$-th layer in the generator $G$.
%
Given an input sketch $S$ and the corresponding deformed sketch $\tilde{S}$, we compute the generator feature matching loss as:

\begin{equation}
	\label{eqn:loss_GFM}
	\mathcal{L}_{GFM}(G)=\mathbb{E}_{S\sim p_{data}(S)} \frac{1}{N_Q} \sum_{q\in Q}  \frac{1}{|G_q|} \|G_q(S)-G_q(\tilde{S}) \|_1,
\end{equation}
%
where $|G_q|$ denotes the number of elements in $G_q(\cdot)$, $Q$ indicates a set of the selected generator layers for computing this loss and the size of $Q$ is $N_Q$. 
We select \td{the xxx layers} of the generator in our experiments.

Besides the generator feature matching loss $\mathcal{L}_{GFM}(G)$, for generator $G$ and multi-scale discriminator $D={D_k | k=1,2,...,N_D}$, the adversarial loss $\mathcal{L}_{GAN}(G, D)$ and the discriminator feature matching loss $\mathcal{L}_{DFM}(G, D)$ are computed as the same form as those in pix2pixHD~\cite{pix2pixHD}. \td{Discussion: add equations of these two losses or not}
%
The objective of the proposed model is:

\begin{equation}
	\label{eqn:new_minmax_game}
	\min_G \max_{D} \mathcal{L}_{GAN}(G, D)+\lambda \mathcal{L}_{DFM}(G, D) +\mu \mathcal{L}_{GFM}(G).
\end{equation}

where $\lambda$ and $\mu$ are the weights for balancing different losses. We set $\lambda=\td{xxx}$ and $\mu=\td{xxx}$ in our experiments.