
\documentclass[sigconf,anonymous,review]{acmart}
\acmSubmissionID{1570}

\usepackage{booktabs} % For formal tables
\usepackage{color}
 % to balance the end of two columns
\usepackage{balance}
\usepackage{bm}
% TOG prefers author-name bib system with square brackets
%\citestyle{acmauthoryear}
%\setcitestyle{nosort,square} % nosort to allow for manual chronological ordering



\usepackage[ruled]{algorithm2e} % For algorithms
\renewcommand{\algorithmcfname}{ALGORITHM}
\SetAlFnt{\small}
\SetAlCapFnt{\small}
\SetAlCapNameFnt{\small}
\SetAlCapHSkip{0pt}

% Metadata Information
%\acmJournal{TOG}
%\acmVolume{38}
%\acmNumber{4}
%\acmArticle{39}
%\acmYear{2019}
%\acmMonth{7}

% Copyright
%\setcopyright{acmcopyright}
%\setcopyright{acmlicensed}
%\setcopyright{rightsretained}
%\setcopyright{usgov}
%\setcopyright{usgovmixed}
%\setcopyright{cagov}
%\setcopyright{cagovmixed}

% DOI
%\acmDOI{0000001.0000001_2}

% Paper history
%\received{February 2007}
%\received{March 2009}
%\received[final version]{June 2009}
%\received[accepted]{July 2009}

\setcopyright{acmcopyright}
\copyrightyear{2020}
\acmYear{2020}
\acmDOI{10.1145/1122445.1122456}

%% These commands are for a PROCEEDINGS abstract or paper.
\acmConference[MM '20]{Proceedings of the 27th ACM International Conference on Multimedia}{October 21--25, 2020}{Seattle, US}
\acmBooktitle{Proceedings of the 27th ACM International Conference on Multimedia (MM '20), October 21--25, 2020, Seattle, US}
\acmPrice{15.00}
\acmISBN{978-1-4503-XXXX-X/20/06}

\newcommand{\cxj}[1]{\textcolor{red}{(#1)}}
\newcommand{\td}[1]{\textcolor{blue}{#1}}
\newcommand{\rmv}[1]{\textcolor{green}{#1}}
\newcommand{\real}{{\mathbb{R}}}
\newcommand{\synS}{\mathcal{S}_{syn}}
\newcommand{\dfmS}{\mathcal{S}_{dfm}}
\newcommand{\hdS}{\mathcal{S}}
\newcommand{\dlt}[1]{}


% Document starts
\begin{document}
% Title portion
%\title{FaceSketching: Interactive Realistic Face Image Creation from Free-hand Sketches }
\title{DeepFacePencil: Creating Face Images from Freehand Sketches}

\begin{abstract}
In this paper, we explore the task of generating photo-realistic face images from hand-drawn sketches. 
%
Existing image-to-image translation methods require a large-scale dataset of paired sketches and images for supervision.
They typically utilize synthesized edge maps of face images as training data. 
However, these synthesized edge maps strictly align with the edges of the corresponding face images, which limit their generalization ability to real hand-drawn sketches with vast stroke diversity. 
To address this problem, we propose DeepFacePencil, an effective tool that is able to generate photo-realistic face images from hand-drawn sketches, based on a novel dual generator image translation network during training. 
%
A novel spatial attention pooling (SAP) is designed to adaptively handle stroke distortions which are spatially varying to support various stroke styles and different level of details.
We conduct extensive experiments and the results demonstrate the superiority of our model over existing methods on both image quality and model generalization to handdrawn sketches.
\end{abstract}

% The code below should be generated by the tool at
% http://dl.acm.org/ccs.cfm
% Please copy and paste the code instead of the example below.
%
\begin{CCSXML}
	<ccs2012>
	<concept>
	<concept_id>10010147.10010257.10010293.10010294</concept_id>
	<concept_desc>Computing methodologies~Neural networks</concept_desc>
	<concept_significance>500</concept_significance>
	</concept>
	</ccs2012>
\end{CCSXML}

\ccsdesc[500]{Computing methodologies~Neural networks}

%\keywords{spatial attention; conditional generative adversarial nets; face; sketch; realistic images}

%
% End generated code
%


\keywords{Image synthesis, spatial attention, sketch-based interface,
face editing, conditional generative adversarial networks}

\dlt{
\begin{teaserfigure}
	\includegraphics[width=\textwidth]{figs/teaser.png}
	\caption{This is a teaser}
	\label{fig:teaser}
\end{teaserfigure}
}

\maketitle


\section{Introduction}


Flexibly creating new content is one of the most important goals in both computer graphics and computer-human interaction. While sketching is an efficient and natural way for common users to express their ideas for designing and editing new content, sketch-based interaction techniques have been extensively studied~\cite{SutherlandSketchPad64,Zeleznik-Sketch96,Igarashi-teddy99,Chen_sketchingreality08,Chen09sketch2photo}. 
Imagery content is the most ubiquitous media with a large variety of display devices everywhere in our daily life. 
Creating new imagery content is one way to show people's creativity and communicate smart ideas.
%
In this paper, we target portrait imagery, which is inextricably bound to our life, and present a sketch-based system, \emph{DeepFacePencil}, which allows common users to create new face imagery by specifying the desired facial shapes via free-hand sketches. 

%%%%%%%%%%%%%%%%%% DL %%%%%%%%%%%%%%%%%%%%%%%%%%%%%
Deep learning techniques have brought significant improvement on the realism of virtual images. 
Recently, a large amount of studies have been conducted on general image-to-image translation which aims to translate an image in one domain to a corresponding image in another domain, preserving the same content, such as structure, scene or objects~\cite{pix2pix,pix2pixHD,CycleGANs,DiscoGANs, DualGANs,BicycleGANs}. 
%
Treating sketches as the source domain and realistic face images as the target domain, \td{this task is a typical image-to-image translation problem.
However, exiting image-to-image translation techniques are not off-the-shelf for this task due to the underlying challenges: data scarcity in the sketch domain and ambiguity in freehand sketches. }


Since there exists no large-scale dataset of \td{paired sketch and face images?} and collecting hand-drawn sketches is time-consuming, existing methods~\cite{pix2pix, pix2pixHD, Lines2Face} utilize edge maps or contours of real face images as training data when applied on the sketch-to-face task. Edge maps and contours enable existing models to be trained in a supervised manner and obtain plausible results on synthesized edge maps or contours . 
%
However, models trained on synthesized data are not able to achieve satisfactory results on hand-drawn sketches, specially on those drawn by common users without considerable drawing skills. 

Since strokes in edge maps and contours align perfectly with edges of the corresponding real images, models trained on edge-aligned data tend to generate unreal shapes of facial parts following the inaccurate strokes when the input sketch is poor-drawn. Hence, for an imperfect hand-drawn sketch, it is a trade-off between \textit{the realism} of the synthesized image and \textit{the correspondence} between input sketch and the edges of the synthesized image.
Models with high edge-alignment fails to be generalized to sketches with imperfect strokes.

Moreover, we observe that the balance between the trade-off mentioned above varies from one position to another across the image. In a portrait sketch, some facial parts might be well-drawn while the others not. For the well-drawn facial parts, the balance are supposed to move towards the correspondence ensuring those parts in synthesized image depicting the user's imagination. On the other hand, the areas of poorly-drawn parts should emphasize the realism and not follow the irregular shapes and strokes.

%%%%%%%%%%%%%%%%%% Our Idea $$$$$$$$$$$$$$$$$$$$$$$
Based on the discussion above, we propose a novel sketch-based synthesis framework which is robust to hand-drawn sketches. A new module, named spatial attention pooling (SAP), is designed to adaptively adjust the spatially varying balance between \textit{realism} and \textit{correspondence} \cxj{alignment?} across the image. In order to break the edge-alignment between sketches and real images, our SAP relaxes strokes with one-pixel widths to multiple-pixel widths using pooling operators. A larger width of a stroke, which is controlled by the kernel size of pooling operator, indicates the less restrict between this stroke and the corresponding edge in the synthesized image. However, the kernel size is not trainable using back propagation algorithm. Hence, for an input sketch, multiple branches of pooling operators with different kernel sizes are added in SAP to get multiple relaxed sketches with different widths. The relaxed sketches are then fused by a spatial attention layer which adjusts the balance of \textit{realism} and \textit{correspondence}. For different location in a portrait sketch, the spatial attention layer assigns high attention to the relaxed sketch with large width if this position requires more \textit{realism} than \textit{correspondence}. 

In summary, our contribution in this paper is three-fold.
\begin{itemize}
\item Based on comprehensive analysis on the edge alignment issue in image translation frameworks, we propose a sketch-to-face translation system that is robust to hand-drawn sketches \td{with various drawing skills}. 
\item A novel deep neural network module for sketch, named \emph{spatial attention pooling}, is designed to adaptively adjust the spatial-variant balance between the realism of the synthesized image and the correspondence between the input sketch and the synthesized image.
\item Extensive experiments demonstrate the superiority of our model over existing methods on perceptual realism and generalization on the sketch-to-image task.
\end{itemize}
\section{Related Work}

\cxj{Image Translation}



\cxj{Face Generation and Editing}

\td{All existing methods for local face editing require the user to provide masks and strokes manually}. In comparison, strokes are only the input to indicate the desired shape. Our system automatically interprets the intended edit and produces the local change accurately. This greatly reduces users' burden and preserves the fluency of user interaction. 

\td{Only local edit in a relatively small area is supported.} As reported in SC-FEGAN~\cite{}, artifacts appear when complete a large region in FaceShop~\cite{}.

\section{Deep Network for Sketch-Photo Translation}
\label{sec:network}

\subsection{Edge alignment in baseline Model}

Generating a realistic photo from sketch can be considered as an image translation problem. 
We use the state-of-the-art Pix2PixHd~\cite{} as our baseline. 
% 
 
We use the CelebA-HD dataset~\cite{} which contains \td{xxx face images in WxH}. All the face images are globally aligned according to their eye positions. 
For each photo, we generate sketch samples \td{by xxx method}.
\td{XX for training and xx for testing.}
%
Using this paired sketch-photo dataset, \td{we trained our method and other state-of-the-art approaches for comparison}.

\subsubsection{Global alignment}
While all the face images in the CelebA dataset \cite{} are globally aligned with their eye positions, the learned generator implicitly embeds the global layout. Once the drawn sketches deviate from this implicitly embedded layout, the learned translation model generate awkward results, as Fig.~\ref{fig:global-align-fail} shows. 


\begin{figure}
	\centering
	\vspace{1.0cm}
	\caption{Generated face images from a sketch that does not follow the aligned face layout.}
	\label{fig:global-align-fail}
\end{figure}

\subsubsection{Data augmentation with geometric translation}
A straightforward way is to augment the training set by random geometric transformation of input sketches. 
However, large interval/range of the transform yields un-convergence of the network training. 
We limit the transformation range to $(-\frac{\pi}{10},\frac{\pi}{10})$ rotation, $(-\frac{W}{20},\frac{W}{20})$ translation, and \cxj{scale?} in order to increase the tolerance of the trained model on the distortion and roughness of hand-drawn sketches. 
%
Moreover, as an interactive system, we also simply provide a reference sketch, like ShadowDrawing~\cite{}.
We place an averaged face contour image on the canvas to provide a reference coordinate system for user to draw their strokes. 
Therefore, the drawn sketches under this geometric reference will be located in or close to space of the training sketches.

\subsubsection{Data augmentation with sketch dilation}

\subsubsection{pix2pixHD without instance normalization}
The baseline model, as well as many existing stylization DNNs, uses spatial normalization to extract style statistics, treating them as spatially uniform on the entire image. 
However, the shallow convolution layers typically extract texture or brightness statistics, which are varying dramastically on different local regions of a drand-drawn sketch. For example, there might be a large number strokes around hair regions or mouth regions to describe details. These heavy strokes bring significant difference while instance normalization at each convolution layer normalizes local patch features with a global factor. Therefore, the extracted features for identical eye strokes could be very different. 

We think that strokes mainly describe the shape features, without little texture information. 

\subsubsection{Spatial Attention Pooling}
When the input hand-drawn sketch is not well-drawn, it is a trade-off between the realism of the output face image and the alignment between the input sketch and the output face image.
%
In order to alleviate the edge alignment between the input sketch and the output face image, we should relax the sharp sketch lines with one-pixel width.
One of the straightforward ways is to smooth the lines of sketches using image smoothing algorithm. 
Another is to dilate the sketch lines so that the widths of lines are of multiple pixels~\cite{DeepSurgery}.
However, the capacity of either the two hand-craft ways above is limited. Because the smoothness and the dilate radius are the same for all positions of the whole image.
%
We argue that the balance between the realism of the output face image and the alignment between the input sketch and the output face image differs from one position to another across the face image. Therefore, the smoothness or the dilation radius should be spatial-specific. 

Based on the discussion above, we propose a new module, called spatial attention pooling (SAP), to dilate the input sketch in a spacial-specific way. Let $\mathbf{r}=\{r_i | i=1,2,...,N_r\}$ be a set of dilation radius. Given an input sketch $s\in \real^{H\times W}$, we first pass it through $N_r$ pooling branches with kernel sizes of $\mathbf{r}$ to get $\{P_{r_i}(s) | i=1,2,...,N_r\}$. Then we compute the spatial attention map $W\in \real^{N_r\times H\times W} $with $W = Softmax(f(s))$, where $f()$ is implemented with two convolutional layers. A softmax layer which is computed over channels is added at the end of the convolutional layers, ensuring that for each position, the sum of weights of all channels equals to $1$. The output of SAP is computed as:
	
\begin{equation}
	SAP(s)=\sum_{i=1}^{N_r} W_i * P_{r_i}(s),
\end{equation}

where $W_i$ is the $i$th channel of $W$.





\paragraph{Sketch Data Generation}
Four options to generate sketches: Canny, HED, AutoTrace [Web..], and Contour. 

. 
\section{Experiments and Discussions}
\label{sec:experiments}

%We propose a novel sketch-to-face translation model which is robust to hand-drawn sketches. 
Our method is robust to hand-drawn sketches. We conduct extensive experiments to demonstrate the effectiveness of our model in generating high-quality realistic face image from sketches drawn by different users with diverse painting skills.  


\subsection{Implementation Settings}
Before showing experimental results, we first introduce details in our network implementation and training. 


\paragraph{Implementation Details}
We implement our model on Pytorch~\cite{Pytorch}. Both generators for edge-aligned sketches and deformed sketches share an encoder-residual-decoder structure with shared weights except that an SAP module is added to the front of the main generator $G_m$ for deformed sketches. 
The encoder consists of four convolutional layers with $2\times$ downsampling, while the decoder consist of four convolutional layers with $2\times$ upsampling. 
Nine residual blocks between the encoder and decoder enlarge the capacity of the generators. 
%Weights of residual blocks and decoders of two generators share with each other. 
The multi-scale discriminator $D$ consists of three sub-networks for three scales separately, same as Pix2PixHD~\cite{pix2pixHD}. 
Instance normalization~\cite{IN} is applied after the convolutional layers to stabilize training. 
ReLU is used as activation for generators and LeakyReLU for discriminator. 
%
\paragraph{Data}
To produce triplets, consisted of sketches, deform sketches, and real images, $(S, S', x)$ for our network training, we use 
CelebA-HQ~\cite{PGGAN}, a large-scale face image dataset which contains 30K $1024\times1024$ high-resolution face images. 
%
CelebAMask-HQ~\cite{CelebAMask-HQ} offers manually-annotated face semantic masks for CelebA-HQ with 19 classes including all facial components and accessories such as skin, nose, eyes, eyebrows, ears, mouth, lip, hair, hat, eyeglass, earring, necklace, neck, and cloth. We utilize semantic masks in this dataset to extract semantic boundary maps as edge-aligned sketches. 
Deformed sketches are generated by vectorizing and adding random offsets to edge-aligned sketches as discussed in the last section.
Real images, sketches, and deformed sketches are resized to $256\times256$ in our experiments.


\paragraph{Training Details}
All the networks are trained by Adam optimizer~\cite{Adam} with $\beta_1=0.5$ and $\beta_2=0.999$. 
For each training stage, the initial learning rate is set to $0.0002$ and starts to decay at the half of training procedure. 
We set batch size as 32. 
The entire training process takes about three days on four NVIDIA GTX 1080Ti GPUs with 11GB GPU memory.



\paragraph{Baseline Model} 
Pix2pixHD~\cite{pix2pixHD} is a state-of-the-art image-to-image translation model for high-resolution images. 
With the edge-aligned sketches and real face images, we train pix2pixHD with its low-resolution version of generator (`global generator') as a baseline model in our experiment, denoted as \textit{baseline}. 
In order to conduct a fair comparison on generalization, we also train the baseline model with augmented dataset by adding pairs of deformed sketches and images, denoted as \textit{baseline\_deform}.
%
The key idea of our method is using SAP and dual generators to improve the tolerance to sketch distortions. The local enhancer part of pix2pixHD, which is designed for high-resolution image synthesis, can be easily added to improve fine textures for both baseline models and our model in the future. 


\subsection{Quantitative Evaluation on Image Quality}

\subsubsection{Evaluation Metrics} 
Evaluating the performance of generative models has been studied for a period of time in image generation literature.
It is proven to be a complicated task because a model with good performance with respect to one criterion does not necessarily imply good performance with respect to another criterion~\cite{GANs_equal}. 
A proper evaluation metrics should be able to present the joint statistics between conditional input samples and generated images.
Traditional metrics, such as pixel-wise mean-squared error, can not effectively measure the performance of generative models. 
We utilize two popular quantitative perceptual evaluation metrics based on image features extracted by DNNs: Inception Score (IS)~\cite{Improved_Techniques} and Fréchet Inception Distance (FID)~\cite{FID}.
These metrics are proven to be consistent with human perception in assessing the realism of images.

\paragraph{Inception Score (IS)}
IS applies an Inception model pre-trained on ImageNet to extract features of generated images and computes the KL divergence between the conditional class distribution and the marginal class distribution. Higher IS presents higher quality of generative images.
Note that IS is reported to be biased in some cases because its evaluation is based more on the recognizability rather than on the realism of generated samples~\cite{evaluation}. 

\paragraph{Fréchet Inception Distance (FID)}
FID is a recently proposed evaluation metric for generative models and proven to be consistent with human perception in assessing the realism of generated images. FID computes Wasserstein-2 distance between features of generated images and real images which are extracted by a pre-trained Inception model. Lower FID indicates that the distribution of generated data is closer to the distribution of real samples.

%\paragraph{Kernel Inception Distance (KID)}
%KID measures the distance of two distributions by calculating the squared maximum mean discrepancy between Inception features. KID is pointed out as an unbiased estimator with a cubic kernel~\cite{KID}. Lower KIDs indicate better generative models.

%\subsection{Generative Quality with Image Translation Networks}


\subsubsection{Image Quality Comparison}
Existing image-to-image translation models can be trained for sketch-to-face translation using paired sketch and image data. 
Since the quality of generated images presents the basic performance of a generative model, we first compare the quality of generated images by different generative models using IS and FID. 
We test these models with edge-aligned sketches that are synthesized from images in the test set. 
Besides the baseline model, we also test Pix2pix~\cite{pix2pix}, which is the first general image-to-image translation framework. It can be applied to a variety of applications by switching the training data. 
We use the default setting to train Pix2pix model with paired edge-aligned sketches and real face images. 
All these methods produce face image in dimension of $256\times256$.

Table~\ref{tab:generative_quality} shows the quantitative evaluation results of four models. Our model surpasses other models by a small margin with respect to two evaluation metrics. 
Visual results are shown in Figure~\ref{fig:generative_quality}. As we can see, all of the four models are able to generate plausible face images from sketches of test dataset. Since these test sketches are generated using the same method as training sketches, the test data distribution is quite close to/overlapped with the training data distribution. Both our model and existing models are easily to be generalize to handle these samples. We will show our model's superiority on generalization to more challenging sketches in the next few experiments.



\begin{table}[h]
	\centering	
	\caption{Results of generative quality comparison.}
	\begin{tabular}{|c|c|c|c|c|}\hline
		& Pix2pix \cite{pix2pix} & Baseline~\cite{pix2pixHD} & Baseline\_deform & Ours \\\hline
		IS & $2.186$ & $2.298$ & $2.369$ & $\textbf{2.411}$\\\hline
		FID & $289.3$ & $259.1$ & $244.3$ & $\textbf{242.1}$\\\hline
	\end{tabular}
	\label{tab:generative_quality}
\end{table} 

\begin{figure}
	\includegraphics[width=0.9\linewidth]{figs/results}
	\caption{Both our model and existing models, which generate plausible photo-realistic face images from sketches in test set, are able to be generalized to test sketches from similar distribution as the training distribution. We will show our model's superiority on generalization to more challenging sketches in the next few experiments.}
	\label{fig:generative_quality}
\end{figure}

\subsection{Comparison of Generalization Capability}

In order to verify the generalization ability of our model, we compare our model with state-of-the-art image translation models by testing with synthesized sketches of different levels of deformation, well-drawn sketches and poorly-drawn sketches by novel users.

\subsubsection{Different Level of Deformation}
As mentioned in Subsection~\ref{subsec:algorithm_data}, we deform an edge-aligned sketch $S$ to obtain a corresponding deformed sketch $S'$ by adding random offsets to the control points and end points of the vectorized strokes in $S$. 
The maximum offset $d$ is set to $11$ in the training data. 
We further create more deformed sketches with multiple levels of deformation, denoted as $S'_d$, by modifying the maximum offset $d$.
%
We examine the generalization ability of our model and baseline model on these deformed sketches. 
Note that the \textit{baseline} is trained with only edge-aligned sketches while our model and \textit{baseline\_deform} model are trained both edge-aligned sketches and deformed sketches with $d=11$.

In this experiment, the input sketches are deformed by larger offsets where the maximum $d$ is set to $30$. 
As shown in Figure~\ref{fig:generalization_examples}, strokes in sketches with larger deformation looks quite different from those in the training sketches including edge-aligned sketches $S$ and deformed sketches $S_{11}'$. 
%
By adding deformed sketches into training data, \textit{baseline\_deform} produces better images than \textit{baseline}. However, when larger deformation occurs, \textit{baseline\_deform} suffers from artifacts in facial features, for example, the mouth in the first example, the eyes of the third and the last case in Figure~\ref{fig:generalization_examples}, 
In comparison, our model produces more realistic face images with more symmetric eyes and find textures, benefited from its ability of capturing shape feature from deformed strokes by our spatial attention pooling module. 
\begin{figure}
	\includegraphics[width=0.9\linewidth]{figs/generalization_examples}
	\caption{For sketches with large deformation, both \textit{baseline} model and \textit{baseline_deform} model fail to generate satisfying results, in which artifacts can be found in areas with large sketch deformation. Our results maintain high quality even large deformation occurs.  }
	\label{fig:generalization_examples}
\end{figure}

\subsubsection{Hand-Drawn Sketches}
Besides the synthesized sketches with stroke deformation, we further examine the model generalization ability by comparing performances of our model with baseline model tested with two kinds of hand-drawn sketches: expert-drawn sketches and common-user sketches drawn by users without professional painting skills.

\paragraph{Expert-Drawn Sketches}
We invite an expert with well-trained drawing skills to draw portrait sketches for testing. 
These expert sketches were drawn on a pen tablet so that the strokes are smooth and precise. 
Note that shading strokes are not drawn. 
%
Figure~\ref{fig:expert_sketches} shows a group of face images generated by different models from several expert-drawn sketches. 
%
Even with well-drawn strokes, the \textit{baseline} and \textit{baseline\_deform} frequently fail to produce realistic textures and complete structures of eyes or mouths. 
%
In comparison, our results are more realistic with fine textures and intact structures. 

\begin{figure}
	\includegraphics[width=0.9\linewidth]{figs/expertsketches}
	\caption{Our model is able to be generalized to well-drawn expert sketches while results of baseline models are degenerated.}
	\label{fig:expert_sketches}
\end{figure}


\paragraph{Common Sketches}
We also invited a large number of graduate students without drawing skills to draw freehand sketches of their imagined faces using mouses. Hence, strokes of these common sketches rough depicts the desired face structure and shapes of facial features with some distortion. 
Moreover, common sketches turn to be of different levels of details. For example, some sketches contains many strokes inside the hair areas which are blank in the training sketches. 
%
Results shown in Figure~\ref{fig:common_sketches} demonstrate that our model is robust to these poorly-drawn sketches. 
In constrast, the diversity of stroke styles and detail levels significantly damage the quality of the results from the \textit{baseline} model and \textit{baseline\_deform} model.


\begin{figure}
	\includegraphics[width=0.9\linewidth]{figs/commonsketches}
	\caption{For these challenging sketches drawn by common users, our model still generate plausible results. Results of baseline models are over-blurred and contains obvious artifacts.}
	\label{fig:common_sketches}
\end{figure}




\section{Conclusion}
In this paper, we present DeepFacePencil, a novel deep neural network which allows common users to create photorealistic by freehand sketching.
%
The robustness of our sketch-based face generator comes from the proposed dual generators and the spatial attention pooling module. 
%
The proposed spatial attention pooling module adaptively adjusts the spatially varying balance between the image realism and the conformance between the input sketch and the synthesized image. 
%
By adding the SAP module to our dual-generator network and training the two generators simultaneously to enforce the main generator to effectively capture face structure and facial feature shapes from coarsely drawn sketches. 
Extensive experiments demonstrate that our DeepFacePencil successfully produce high quality face images from freehand sketches drawn by users in diverse drawing skills.
%



% Bibliography
\bibliographystyle{ACM-Reference-Format}
\balance 
\bibliography{sketch}
%\bibliography{sketchRef}



\end{document}
