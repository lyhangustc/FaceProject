% !TeX root = SketchFace.tex
% 

The task of sketch-to-photo translation can be defined as looking for a generator $G(S)$ so that the generated image $x=G(S)$ from a hand-drawn sketch $S$ looks realistic and keeps the shape characteristics for the input sketch.
%
Existing image translation techniques train a neural network as the generator with paired of sketch and photo data $(\mathcal{S}, \mathcal{X})$.
%
Due to the scarcity of real hand-drawn sketches, existing techniques synthesize sketches in a certain style to approximate the sketch set $\mathcal{S}$ from face image set $\mathcal{X}$ to train their generator in an adversarial manner.
The synthesized sketches $\mathcal{S}_{syn}$ are usually well aligned with the face images and present different distributions from hand-drawn sketches $\mathcal{S}$.
These models typically fail to generalize to hand-drawn sketches by common users. 
%


As shown in Figure~\ref{fig:architecture}, $G_m$ is the main generator trained by deformed sketches $S'$, aiming to generate plausible photo-realistic face images from unseen hand-drawn sketches in test stage. 
$G_a$ is an auxiliary generator trained with edge-aligned sketches whose goal is to guide $G_m$ to adaptively sense the line distortion in deformed sketches.


We propose a novel network architecture with a specially designed training strategy to improve the capability of the sketch-based image generator.
%
Figure~\ref{fig:architecture} shows the overview of our method.
%
In order to synthesize a set of sketches $\synS$ that has similar distribution with hand-drawn sketches $\hdS$, we deform the edge-aligned sketches to generate a set of deformed sketches $\dfmS$ to augment the training set.
We propose a novel framework using dual generators from the edge-aligned sketches $\synS$ and the deformed sketches $\dfmS$ respectively.
%
$G_m$ is the main generator trained by deformed sketches $S'$, aiming to generate plausible photo-realistic face images from unseen hand-drawn sketches in test stage. 
$G_a$ is an auxiliary generator trained with edge-aligned sketches whose goal is to guide $G_m$ to adaptively sense the line distortion in deformed sketches.
%
A spatial attention pooling module (SAP) is added before the encoder $E_m$ of $G_m$ to adjust the spatially varying balance between \textit{the realism} of generated images and \textit{the conformance} between the generated image and the input sketch. 
%

The dual generators are trained together under a set of supervision. 
First, for a triplet $(S,S',x)$, both the generators $G_m$ and $G_a$ are trained to produce images $G_m(S')$ and $G_a(S)$ to approximate the real image $x$ under a reconstruction loss $L_{rec}$. 
Second, a multi-scale discriminators $D$ is employed for three combinations of sketch-image pairs to distinguish real face images from generated fake images in both global and local scales. 
By training the dual generators simultaneously with the discriminator adversarially, our main generator $G_m$ effectively captures the spatially-varying stroke distortions and maps it to the manifold of well-drawn sketches to produce realistic face images.


\dlt{

%Both generators are encoder-residual-decoder architectures, including an encoder, several residual blocks and a decoder, which is proven to be effective for generate high-quality images. 
%The generators share weights of the residual blocks and decoders since the high-level features of both generators are supposed to share with each other.



The discriminator is a multi-scale discriminator~\cite{pix2pixHD} which distinguishes real face images from generated fake images in both global and local scales.
For the discriminator, the generated image $G_m(S')$ and $G_A(S)$ concatenated with their input sketch $S$ and $S'$ respectively are treated as fake samples while a real face image sampled from real face distribution concatenated with its corresponding sketch is regarded as a real sample. 
%
%In the adversarial training stage, the generator and the discriminator are updated alternately.

In order to guide the model with SAP to be tolerant with line distortion of deformed sketches, we design a novel generator feature matching loss for our task, besides the adversarial loss, the reconstruction loss and the discriminator feature matching loss. The model is trained in a multi-stage training schedule to ensure the convergence of the training.
%



We arrange this section as follow. We first introduce the dataset we construct for our model in Subsection~\ref{subsec:algorithm_data}.
Then we describe the architecture of our model in Subsection~\ref{subsec:algorithm_overview} and the proposed SAP in Subsection~\ref{subsec:algorithm_sap}.
At last we discuss losses applied in our model in Subsection~\ref{subsec:algorithm_loss} and the multi-stage training schedule in Subsection~\ref{subsec:algorithm_training}.
}
