% !TeX root = SketchFace.tex

 
A sketch-to-image model trained with edge-aligned sketch-image pairs tends to generate images whose edges strictly align with the stokes of the input sketch.
When an input hand-drawn sketch is not well-drawn, line distortions in the input sketch damages the quality of the generated face image. 
It is a trade-off between the realism of the generated face image and the correspondence between the input sketch and the output face image.
%
In order to alleviate the edge alignment between the input sketch and the output face image, we propose to relax thin strokes to a tolerance region with various width.
%
A straightforward way is to smooth the strokes to multi-pixel width by image smoothing or dilation. 
%
However, the capacity of this hand-crafted way is limited, because the uniform smoothness for all positions of the whole sketch violate the unevenness of hand-drawn sketches on depicting different facial parts. 
%
We argue that the balance between the realism and the correspondence differs from one position to another across the face image. Therefore, the relaxation degree should be spatially varying. 


Based on the discussion above, we propose a new module, called spatial attention pooling (SAP), to adaptively relax the strokes in the input sketch to spatially varying tolerance regions. 
%
A stroke with a larger width indicates the less restrict between this stroke and the corresponding edge in the synthesized image. The widths are controlled by the kernel sizes of pooling operators. However, the kernel size of pooling operator is not trainable using back propagation algorithm. SAP applies multiple branches of pooling operators with different kernel sizes to get multiple relaxed sketches with different widths. The relaxed sketches are then fused by a spatial attention layer which spatially adjusts the balance of \textit{realism} and \textit{correspondence}. The module is formulated as follow.

%Let $\mathbf{r}=\{r_i | i=1,2,...,N_r\}$ be a set of relax radius. 
The architecture of SAP is shown in Figure~\ref{fig:sap}.
Given an input deformed sketch $S'\in \real^{H\times W}$, we first pass it through $N_r$ pooling branches with different kernel sizes of $\{r_i, i=1,\ldots, N_r\}$ to get $\P_{i}=Pooling_{r_i}(S') (i=1,\ldots,N_r)$. 
Then we utilize convolutional layers to extract feature maps of $P_i$ separately. These feature maps are concatenated to get a relaxed representation of $S'$, denoted as $R$:
%
\begin{equation}
R=Cat(Conv_1(P_1), Conv_2(P_2),..., Conv(P_{N_r})),
\end{equation}
where $Conv_i() (i=1,\ldots,N_r)$ indicates convolutional layers, $Cat$ is a channel-wise concatenate operator.

On the other hand, we compute a spatial attention map $A$ which controls the relax degrees of all positions by assigning different attention weights to $R$.
A stoke with a large distortion is supposed to be assigned with a large relax degree. Hence, $A$ is supposed to adaptively pay more attention (a large weight) to a $Conv_i(P_i)$ with a large kernel size in the areas with large line distortions.
%
A straightforward way to get $A$ is passing the input sketch through a few convolutional layers and these convolutional layers are trained to detect the areas with line distortions. However, we found the a few convolutional layers are insufficient to learn to detect line distortions directly. Therefore, we introduce a two-class classifier to ease the detection. Specifically, we pre-train a fully-convolutional two-class classifier $C$ with three convolutional layers to distinguish sketches from deformed sketches. Then we utilize this pre-trained classifier to extract features of the input sketch $S$ to get ${C_i(S), i=1,2,3}$, where $C_i(·)$ denotes the $ith$ feature maps extracted by $C$. These feature maps from classifier emphasize the differences between sketches and deform sketches. We resize and concatenate these feature maps, and pass them through three convolutional layers to get the spatial attention map:
%	
\begin{equation}
A=Softmax(Conv([C_1, Up_2(C_2), Up_4(C_3)])), 
\end{equation}
%
where $Up_2$ and $Up_4$ indicates $2\times$ and $4\times$ upsampling, Conv(·) indicates three cascaded convoutional layers, and $Softmax(·)$ is a softmax layer computed over channels to ensuring that for each position of $A$, the sum of weights of all channels equals to $1$.

At last, the output SAP is computed as:
%	
\begin{equation}
SAP(S')=A*R,
\end{equation}
%
where $*$ is element-wise multiplication.

%






